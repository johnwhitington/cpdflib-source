\documentclass[a4paper]{memoir}
\usepackage{palatino}
\usepackage{listings}
%\usepackage{microtype}
\usepackage{graphics}
\usepackage[plainpages=false,pdfpagelabels,pdfborder=0 0 0]{hyperref}
\newcommand{\smallgap}{\vspace{4mm}}
\newcommand{\cpdf}{\texttt{cpdf}}
\addtolength{\textwidth}{20mm}
\lstset{
language=C,
basicstyle=\small\ttfamily,
%numbers=left,
%numberstyle=\tiny,
linewidth=15cm,
frame=tb,
columns=fullflexible,
showstringspaces=false
}

\makeindex
\begin{document}

\frontmatter
\thispagestyle{empty}

\begin{flushright}

{\sffamily \bfseries \Huge Coherent PDF Library (libcpdf)}

\vspace{12mm}

{\Huge Developer's Manual}\\\vspace{2mm}
Version 2.2 (January 2017)

\vspace{25mm}

\vfill

\includegraphics{logo.pdf}

\vspace{2mm}
{\sffamily \bfseries \LARGE Coherent Graphics Ltd}

\end{flushright}

\clearpage

\thispagestyle{empty}
\noindent For bug reports, feature requests and comments, email\\ \texttt{contact@coherentgraphics.co.uk}

\vspace*{\fill}
\noindent\copyright 2017 Coherent Graphics Limited. All rights reserved.

\smallgap 
\noindent Adobe, Acrobat, Adobe PDF, Adobe Reader and PostScript are
registered trademarks of Adobe Systems Incorporated. Windows is a registered trademarks of Microsoft Corporation.

% Letter
\cleardoublepage
\tableofcontents

\cleardoublepage
\mainmatter
\chapterstyle{hangnum}
\pagestyle{ruled}

\chapter*{Note}

The chapters are numbered to be the same as those in the manual for the Coherent PDF Command Line Tools (cpdfmanual.pdf). However, some of the content has moved where circumstances dictate -- for example, encryption is wholly described in Chapter 1 rather than Chapter 4.

All functions in cpdflib take or return UTF8 text - the command line tool's other modes, "Raw" and "Stripped" are not supported.

\chapter*{Installation}
The Coherent PDF Library is provided either in compiled form (the archive file \texttt{libcpdf.a} and the library header \texttt{cpdflibwrapper.h}), or as source code. Instructions for building from source are included in the distribution.

\textbf{Linux, Unix and OS/X} Place \texttt{libcpdf.a, libbigarray.a and libunix.a} somewhere suitable. Instruct your C linker to link with them. Place \texttt{cpdflibwrapper.h} somewhere suitable and instuct your C compiler to search for headers there. \noindent The library is now ready for use.

\textbf{Microsoft Windows} Place \texttt{libcpdf.dll} somewhere suitable and instruct your C compiler to link with it. Place \texttt{cpdflibwrapper.h} somewhere suitable and instuct your C compiler to search for headers there. \noindent The library is now ready for use.

\setcounter{chapter}{-1}
\chapter{Preliminaries}

\begin{small}
\lstinputlisting{splits/c00}
\lstinputlisting{splits/c01}
\end{small}

\chapter{Basics}

\begin{small}
\lstinputlisting{splits/c02}
\end{small}

\chapter{Merging and Splitting}
\begin{small}
\end{small}

\chapter{Pages}
\begin{small}
\end{small}

\chapter{Encryption}
\textit{Covered in Chapter 1.}

\chapter{Compression}
\begin{small}
\end{small}

\chapter{Bookmarks}
\begin{small}
\end{small}

\chapter{Presentations}
\textit{Not supported by libcpdf. Use the command line tools instead.}

\chapter{Logos, Watermarks and Stamps}
\begin{small}
\end{small}

\chapter{Multipage Facilities}

\begin{small}
\end{small}

\chapter{Annotations}
\textit{Not supported in libcpdf. Use the command line tools instead.}

\chapter{Document Information and Metadata}
\begin{small}
\end{small}

\chapter{File Attachments}
\begin{small}
\end{small}

\chapter{Miscellaneous}
\begin{small}
\end{small}

\chapter{Page Labels}
\begin{small}
\end{small}

\appendix
\chapter{Dates}
\label{dates}
\index{dates!defined}
Dates in PDF are specified according to the following format:

\begin{framed}
\texttt{D:YYYYMMDDHHmmSSOHH'mm'}\\\\where:

\begin{itemize}
  \item \texttt{YYYY} is the year;
  \item \texttt{MM} is the month;
  \item \texttt{DD} is the day (01-31);
  \item \texttt{HH} is the hour (00-23);
  \item \texttt{mm} is the minute (00-59);
  \item \texttt{SS} is the second (00-59);
  \item \texttt{O} is the relationship of local time to Universal Time (UT), denoted by '+', '-' or 'Z';
  \item \texttt{HH} is the absolute value of the offset from UT in hours (00-23);
  \item \texttt{mm} is the absolute value of the offset from UT in minutes (00-59).
\end{itemize}
\end{framed}

\noindent A contiguous prefix of the parts above can be used instead, for lower
accuracy dates. For example:

\begin{framed}
   \small\noindent\verb!D:2011! (2011)
   
   \vspace{1.5mm}
   \noindent\verb!D:20110103! (3rd March 2011)

   \vspace{1.5mm}
   \noindent\verb!D:201101031854-08'00'! (3rd March 2011, 6:54PM, US Pacific Standard Time)
   
\end{framed}
\chapter{Example Program in C}
This program loads a file from disk and writes out a document with the original included three times. Note the use of \texttt{cpdf\_startup}, \texttt{cpdf\_lastError} and \texttt{cpdf\_clearError}.

\begin{small}
\begin{verbatim}
#include <stdbool.h>
#include "cpdflibwrapper.h"

int main (int argc, char ** argv)
{
  /* Initialise cpdf */
  cpdf_startup(argv);

  /* Clear the error state */
  cpdf_clearError();

  /* We will take the input hello.pdf and repeat it three times */
  int mergepdf = cpdf_fromFile("hello.pdf", "");

  /* Check the error state */
  if (cpdf_lastError) return 1;

  /* The array of PDFs to merge */
  int pdfs[] = {mergepdf, mergepdf, mergepdf};

  /* Clear the error state */
  cpdf_clearError();

  /* Merge them */
  int merged = cpdf_mergeSimple(pdfs, 3);
  
  /* Check the error state */
  if (cpdf_lastError) return 1;

  /* Clear the error state */
  cpdf_clearError();

  /* Write output */
  cpdf_toFile(merged, "merged.pdf", false, false);

  /* Check the error state */
  if (cpdf_lastError) return 1;

  return 0;
}
\end{verbatim}
\end{small}

\backmatter
%\printindex
\end{document}

