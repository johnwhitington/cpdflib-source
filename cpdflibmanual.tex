\documentclass[a4paper]{memoir}
\usepackage{palatino}
\usepackage{listings}
\usepackage{microtype}
\usepackage{graphics}
\usepackage[plainpages=false,pdfpagelabels,pdfborder=0 0 0]{hyperref}
\newcommand{\smallgap}{\vspace{4mm}}
\newcommand{\cpdf}{\texttt{cpdf}}
\addtolength{\textwidth}{20mm}
\lstset{
language=C,
basicstyle=\small\ttfamily,
%numbers=left,
%numberstyle=\tiny,
frame=tb,
columns=fullflexible,
showstringspaces=false
}

\makeindex
\begin{document}
\frontmatter
\thispagestyle{empty}

\begin{flushright}

{\sffamily \bfseries \Huge Coherent PDF Library (libcpdf)}

\vspace{12mm}

{\Huge Developer's Manual}\\\vspace{2mm}
Version 1.8 (December 2013)

\vspace{25mm}

\vfill

\includegraphics{logo.pdf}

\vspace{2mm}
{\sffamily \bfseries \LARGE Coherent Graphics Ltd}

\end{flushright}

\clearpage

\thispagestyle{empty}
\noindent For bug reports, feature requests and comments, email\\ \texttt{contact@coherentgraphics.co.uk}

\vspace*{\fill}
\noindent\copyright 2013 Coherent Graphics Limited. All rights reserved.

\smallgap 
\noindent Adobe, Acrobat, Adobe PDF, Adobe Reader and PostScript are
registered trademarks of Adobe Systems Incorporated. Windows is a registered trademarks of Microsoft Corporation.

% Letter
\cleardoublepage
\tableofcontents

\cleardoublepage
\mainmatter
\chapterstyle{hangnum}
\pagestyle{ruled}

\chapter*{Notes}

The chapters are numbered to be the same as those in the manual for the Coherent PDF Command Line Tools (cpdfmanual.pdf).

\chapter*{Installation}
The Coherent PDF Library is provided either in compiled form (the archive file \texttt{libcpdf.a} and the library header \texttt{cpdflibwrapper.h}), or as source code. Instructions for building from source are included in the distribution.

Place \texttt{libcpdf.a} somewhere suitable. Instruct your C linker to link with it. Place \texttt{cpdflibwrapper.h} somewhere suitable and instuct your C compiler to search for headers there. \noindent The library is now ready for use.

\chapter{Preliminaries}

\begin{lstlisting}
/* The function cpdf_startup must be called with argv before using the
library. */
void cpdf_startup (char **);

/* Set demo mode. Upon library startup is false. If set, files written will
 * have the text DEMO stamped over each page. This stamping will also slow down
 * the library significantly. */
void cpdf_setDemo(int);

/* Errors. lastError and lastErrorString hold information about the last error
 * to have occurred. They should be consulted after each call. If
 * cpdf_lastError is non-zero, there was an error, and cpdf_lastErrorString
 * gives details. If cpdf_lastError is zero, there was no error on the most
 * recent cpdf call. */
int cpdf_lastError;
char* cpdf_lastErrorString;

/* Clear the current error state. */
void cpdf_clearError (void);

/* A debug function which prints some information about resource usage. This
 * can be used to detect if PDFs or ranges are being deallocated properly. */
void cpdf_onExit (void);

/* Remove a PDF from memory, given its number. */
void cpdf_deletePdf(int);

/* Calling replacePdf(a, b) places PDF b under number a. Original a and b are
 * no longer available. */
void cpdf_replacePdf(int, int);

/* To enumerate the list of currently allocated PDFs, call
 * cpdf_startEnumeratePDFs which gives the number, n, of PDFs allocated, then
 * cpdf_enumeratePDFsInfo and cpdf_enumeratePDFsKey with index numbers from
 * 0...(n - 1). Call cpdf_endEnumeratePDFs to clean up. */
int cpdf_startEnumeratePDFs(void);
int cpdf_enumeratePDFsKey(int);
char* cpdf_enumeratePDFsInfo(int);
void cpdf_endEnumeratePDFs(void);

\end{lstlisting}

\chapter{Basics}
\chapter{Merging and Splitting}
\chapter{Pages}
\chapter{Encryption}
\chapter{Compression}
\chapter{Bookmarks}
\chapter{Presentations}
\chapter{Logos, Watermarks and Stamps}
\chapter{Multipage Facilities}
\chapter{Document Information and Metadata}
\chapter{File Attachments}
\chapter{Miscellaneous}
\chapter{Page Labels}
\chapter{Special functionality 1. -- Encryption and Permission status}
\chapter{Special functionality 2. -- Undo}

\appendix
\chapter{Dates}
\label{dates}
\index{dates!defined}
Dates in PDF are specified according to the following format:

\begin{framed}
\texttt{D:YYYYMMDDHHmmSSOHH'mm'}\\\\where:

\begin{itemize}
  \item \texttt{YYYY} is the year;
  \item \texttt{MM} is the month;
  \item \texttt{DD} is the day (01-31);
  \item \texttt{HH} is the hour (00-23);
  \item \texttt{mm} is the minute (00-59);
  \item \texttt{SS} is the second (00-59);
  \item \texttt{O} is the relationship of local time to Universal Time (UT), denoted by '+', '-' or 'Z';
  \item \texttt{HH} is the absolute value of the offset from UT in hours (00-23);
  \item \texttt{mm} is the absolute value of the offset from UT in minutes (00-59).
\end{itemize}
\end{framed}

\noindent A contiguous prefix of the parts above can be used instead, for lower
accuracy dates. For example:

\begin{framed}
   \small\noindent\verb!D:2011! (2011)
   
   \vspace{1.5mm}
   \noindent\verb!D:20110103! (3rd March 2011)

   \vspace{1.5mm}
   \noindent\verb!D:201101031854-08'00'! (3rd March 2011, 6:54PM, US Pacific Standard Time)
   
\end{framed}
\chapter{Example Program in C}
This program loads a file from disk, adds page numbers, and then writes the file encrypted to disk.
\begin{small}
\begin{verbatim}

\end{verbatim}
\end{small}

\backmatter
\printindex
\end{document}

